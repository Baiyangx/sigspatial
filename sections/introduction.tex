\section{Introduction}
Social media has become an inseparable part of people’s daily life. Especially
when an important incident happens, people tend to express their opinion on
SNS. Using these data for analysis, we can understand how people react to
such news events. In case of pandemic, this will provide a reference for the
development of epidemic prevention measures and evaluation. As for such
sentiment analysis several approaches have been done, and some previous
studies accomplished success by using machine learning technique. Kastrat el
al.(2021)~\cite{kastrati2021deep} used a deep learning based sentiment
analyser called ALBANA is proposed and validated on the collected and curated
COVID-19 dataset. They apply an attention mechanism to characterize the word
level interactions within a local and global context to capture the semantic
meaning of words. Among other logical approaches, there is a study~\cite
{chakraborty2020sentiment} proposes the implementation of fuzzy logic for
taming the fuzziness of sentiments. As fuzzy sets are ideally suited to
counter the ambiguities in life, the authors have proposed the initial
integration of fuzzy logic in effectively handling the sentiment
identification of tweets (Chakraborty et al., 2020). This approach provides
ideas for a more precise study of emotional factors in texts.

In order to more accurately study and even predict the changing trend of
people’s thinking to the epidemic, we use a machine learning and deep
learning model to analyze the sentiment scores of tweets. We develop a binary
classifier peoples' sentiment into positive and negative classes. 

Beyond the classification, such Machine Learning or Deep Learning-based
approaches have been tried in many studies. [11] used a Bidirectional
Transformer based model to analyze sentiments of Indian citizens and compare
the results with existing models. Weighting sentiments between a negative and
positive perspective from the social media posts has been suggested by
[12] who used CNN-based model to score the sentiment between 0 and 1, 1 being
the most negative and 0 being the most positive sentiment in text. False news
spread regarding COVID-19 is a prominent task in epidemic modeling.
[13] proposed a Modified-LSTM based model to detect fake news regarding
COVID-19 which can used to separate the unintentional biases in a COVID-19
data stream and learn the meta features of a falsely claimed COVID-19
sentiment. There is also a study analyzing public response to COVID by
utilizing Twitter data in a purpose of extracting and somewhat monitoring
major public opinion. Based on the machine learning method Latent Dirichlet
Allocation (LDA) with sentiment analysis on a list of COVID-19–related
hashtags as search terms to fetch tweets, Xue et al.(2020) reports several
significant results that the public uses a variety of terms when referring to
COVID-19, including virus, COVID-19, coronavirus, and corona virus, and that
public discussions about the Chinese Communist Party (CCP) and the spread of
the virus emerged as a new topic that was not identified in previous studies,
and so on [15]. Alamoodi et al.(2021) provides a valuable review on different
studies about sentiment analysis on not only the COVID outbreak but also
numerous epidemics such as the Ebola, Middle East respiratory syndrome
(MERS), N1H1 and Zika which are observed for the last several decades in
human history [16]. For classifying the public sentiment into positive,
negative and neutral group, Jalil et al.(2022) adopts the natural language
processing model (NLP) Transformer and its expanded version BERT
(Bidirectional Encoder Representations from Transformers). Although this
study is simply focused on classification task on the sentiment analysis, its
research method hints to our project about the possibility of NLP usage in
this field [10].

Furthermore, as for the spatio-temporal analysis, specifically on using social
media to track the spread of infectious disease, it has been a long time
since some attempts have been made even before COVID19 outbreak. For example,
there is a study to track disease activity about the notorious influenza H1N1
and swine flue in 2010s and related public sentiment [14]. It is interesting
to note that the tweet data collected is time-stamped and geolocated
harnessing the user’s selfdeclared home location, since this method can be
adopted to our project. And [17] proposed a real-time simulated social media
based learning model that utilizes disease progress model to constrain the
temporal model. [18] used 4-deep neural network model and a classical machine
learning model to analyze the sentiment of COVID-19 tweets. A meta learner
learns to predict from the feature outputs of the presiding models.

Until now, the existing studes have focused primarily on the public's
sentiment analysis related to the pandemic, or used it as a case study to
verify the performance of specific machine learning or deep learning models.
There are relatively few studies focused on the pandemic, and very few
studies have examined dynamic changes over time in specific regions. However,
our research is based on spatio-temporal data mining, and by conducting the
sentiment analysis based on Twitter data mining, it is possible to track
changes in emotional patterns accompanying changes in time and location.
Furthermore, we leverage the latest technology in natural language processing
that is the attention mechanism, and the detailed sub topic clustering is
carried out under the big theme of pandemic. Through this, it becomes
possible to analyze people's emotions through detailed sub-topics. Once our
long-term goal is achieved, by doing spatio-temporal data mining-based
emotional analysis and detailed sub-topic clustering techniques at an
in-depth level, we can contribute to deeper and more meaningful sentiment
analysis. 