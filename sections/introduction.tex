\section{Introduction}
\label{sec:intro}
The rampaging COVID-19 pandemic has greatly affected emotion and everyone’s
daily life in the past two years. The general public’s emotions went through
a roller coaster-like up-and-downs: shocked by the deadly known disease,
overwhelmed by misinformation, upset by the mitigation policies, and
regaining confidence with the development of vaccines. Ubiquitous user-input
content on social media and online services have generated a tremendous
amount of information that can reflect the users’ emotion and opinion towards
social events. With the abundance of the generated social media data during
the pandemic, more users intend to express their emotions and opinions about
COVID-19-related social events, such as the mitigation policies or the
invention of the vaccines. Such dramatic social media data increase provides
great research opportunities in social media mining and natural language
processing.  

Topic modeling analysis is one of the target research areas in
social media analysis for COVID-19. In the case of pandemics, topic modeling analysis provides a reference for the development of epidemic
mitigation measures and evaluation techniques. In the context of sentiment
analysis, some previous studies accomplished success by using the machine
learning technique. For instance, previous work~\cite{kastrati2021deep} used
a deep learning-based sentiment analyzer to collect and curate the COVID-19
dataset. Attention mechanisms were applied to characterize the word-level
interactions within a local and global context to capture the semantic
meaning of words. Among other logical approaches, Chakraborty et al.~\cite
{chakraborty2020sentiment} propose the implementation of fuzzy logic for
taming the fuzziness of sentiments. However, the existing works in social
media sentiment analysis lack the consideration of spatial and temporal
factors. Our proposed system provides both spatial and temporal aspects of
sentiment analysis under the topic of COVID-19. 

Topic detection and modeling are the second focus of our proposed system.
Topic detection is an important part of addressing the said problem to
separate points of interest among many candidates. Several noble studies have
been conducted in the last few decades to mine the most appropriate topic
associated with a text phrase. A topic graph-based approach was proposed by
Batool et al.~\cite{batool2013precise} where a topic graph from vectorized
Twitter data was proposed with term frequency as a heuristic and social
relationship between the virtual users. Other work~\cite
{xie2016unsupervised} proposed a time-dependent burst detection technique
that focuses on two and three-word data acceleration as a spread tendency to
make early detection based on the keywords. Abd-Alrazaq et al.~\cite
{abd2020top} used Latent Dirichlet Allocation (LDA) for topic modeling on
English twitter data. Gencoglu et al.~\cite{gencoglu2020large} used a large
Twitter dataset to analyze semantic topic clusters using Language Agnostic
BERT Sentence Embeddings (LaBSE). We combine the advantages of LDA and neural
network models to develop a transformative model for topic discovery and
monitoring components in the system. 

The storyline generation problem was first studied by Kumar et al.~\cite
{kumar2008algorithms} as a generic redescription mining technique, by which a
series of redescription between the given disjoint and dissimilar object sets
and corresponding subsets are discovered. Storytelling is an efficient way to
solve the issue of information overload. By extracting critical and connected
entities, the original document is structurally summarized. Current works
contain two categories: Textual Storytelling~\cite
{kumar2008algorithms,hossain2012storytelling,fang2011rex,voskarides2015learning,lee2012joint,shahaf2012metro}
and Visual Storytelling~\cite
{kim2014joint,park2015expressing,wang2012generating}. A storyline generation
component is deployed in the proposed system. The storyline generation is
capable of summarizing the highlighted COVID-19-related events with social
media data, and a visualization component is also developed to demonstrate
the comparisons between the official releases of the mitigation events and
the identified topic events from social media. 

In this paper, we present the Pandemic Topic Detection and Monitoring system
(PanTop), which is capable of 1) collecting and retrieving social media data
on COVID-19-related topics, 2) detecting hidden trends of topics from social
media posts, and 3) generating and visualizing storylines for the extracted
COVID-19-related topics. This paper is structured as follows: Section~\ref
{sec:method} discusses the architecture of the PanTop system; Section~\ref
{sec:exp} illustrates the experiment and data of the system; Section~\ref
{sec:case} demonstrates the case studies revealed by the PanTop system; in
section~\ref{sec:conclude}, we conclude our discoveries from the proposed
PanTop system. 
