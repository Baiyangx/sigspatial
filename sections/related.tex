\section{Related Works}
\subsection{Pandemic Spread Forecasting}
Since influenza has a great impact on mankind, many studies have been trying
to predict and prevent it. Some of them approach the issue in formal way,
i.e., theoretic and mathematical way, and a representative model in this
category is the Susceptible Infected Removed (SIR) which is a mathematical
modelling of infectious diseases in Epidemiology [1][2]. Arun and Iyer
(2020), based on real-time data from Johns Hopkins Center for Systems Science
and Engineering, provides the COVID transmission analysis-based prediction of
the pandemic scale, the recovery rate and the fatality rate. In particular,
it is noted that the algorithm Rough Set (RS) based Support Vector Machine
(SVM) has a good performance in prediction of the COVID cases by time. Cooper
et al.(2020), also modelling the spread of the epidemic based on the SIR,
presents an updated SIR model. This study tackles on some static assumptions
by the classical SIR, and can consider new epicentres springing up around the
world at different times dynamically, by adopting the three differential
equations named the Ordinary Differential Equations (ODEs). The SIR is a
powerful tool to such research but it has some limitation in that there are
so many unexpected variables such as the latent period of virus, quarantine,
government policy, virus evolution, and etc.

To cope with this issue, there have been a series of studies to refine such
predictions by utilizing the cutting-edge machine learning and deep learning
techniques. There is a study for predicting COVID spread in several regions
based on rather general machine learning model. Wieczorek et al.
(2020) reports a prediction result of pandemic in several countries in the
world based on the Recurrent Neural Network with the same data source above
of the Center for Systems Science and Engineering (CSSE) at Johns Hopkins
University [25]. [26] has proposed a two branch LSTM based deep neural
network model that utilizes synthetic data from pre-COVID-19 statistics to
predict influenza-like disease epidemic in short-term but high resolution
predictive model.

\subsection{Sentiment Analysis}
As for sentiment analysis several approaches have been done, and some previous
studies accomplished success by using machine learning technique. [29] use
three classes investigation on people’s emotions, which are Analytical,
Depressed, Angry [29]. They implemented CNN and LSTM to analyze the emotions
or sentiment of Bangladeshi people in this crisis situations using Deep
learning and found the relatively high accuracy. [28] developed a Sentiment
Analysis framework named Compass to do spatio temporal sentiment analysis on
US Election. This framework facilitates the end user to select an arbitrary
time range to visualize popularity of the two political parties for each
county (or state)of US for the specified time range, which can also have the
potential to be used to show epidemic trends.

Meanwhile, sentiment analysis from social media posts plays a crucial role on
developing sentiment analysis model which can be used to analyze the dynamics
of a pandemic world wide. [11] used a Bidirectional Transformer based model
to analyze sentiments of Indian citizens and compare the results with
existing models.

\subsection{Topic clustering}
Topic detection is an important part of addressing the said problem to
separate points of interest among many candidates. Several noble studies have
been conducted in the last few decades to mine the most appropriate topic
associated with a text phrase - both supervised and unsupervised. A topic
graph-based approach has been proposed by [19] where they find the topic
graph from vectorized twitter data with term frequency as an heuristic and
the social relationship between the virtual users. [20] has proposed a
time-dependent burst detection technique which focuses on two and three word
data acceleration as spread tendency to make early detection based on the
keywords. [21] used Latent Dirichlet Allocation (LDA) for topic modeling on
english twitter data. [23] used large twitter dataset to analyze semantic
topic clusters using Language Agnostic BERT Sentence Embeddings(LaBSE). Their
experiment was performed on three different languages (English, French,
Spanish) and had 10 different categories of annotations such
as "Donate", "News \& Press", "Prevention" etc. From the annotated data they
produced the word embeddings in numerical hyperspace and trained on the
aforementioned classes. Since they had annotated data and pre-selected class
distribution, their performance was good. In [24] the authors have proposed a
multi-stage topic clustering that uses both traditional clustering method and
classification in sequence to generate the final clusters. From each topic
cluster generated from traditional method, they trained using a multi-class
classifier to learn the feature space of individual cluster and their method
then predicted on unseen data based on the training. In [27] authors also
used LDA to cluster the initial topic terms and then used TF-IDF to generate
word vector embeddings and find out the word cloud from each of the topic
cluster. They then compared the sentiment score based on lexicon-based
approach which uses pretrained model.

\subsection{Tweet data Classification}
Rabia Batool et al [3] use keyword based knowledge extraction to analyze the
information from tweets. Moreover, they apply a domain specific seed based
enrichment technique on the enhance of the extracted knowledge. In our
research project, we also focus on the keywords in the tweets. Besides, We
use multiply strategies to exact the keywords based on the part of speech and
visualize classification results. 

Kyosuke Nishida et al [22] publish an efficient way to categorize invisible
tweets. By using data compression, they get to handle multilingual tweets in
the same manner and to effectively utilize the word context of the tweet. Our
research depends only on the English but the idea to extract the word context
is important for us because of the 140 character limit. 

\subsection{Tweet data Visualization}
Yusuke Hara [4] use GPS  data to obtain traffic condition and express  the
spatial  spread  of  people’s  return-home  behaviour in behaviour analysis
of a natural disaster. In this case, the researcher need the extra
information about the public transport and route. Our research have the
similar requirement, We need to show the geographic of states especially the
boundary. Thus, combining the location information from dataset with the
geographic information from other resource is also important. But the
limitation of application of GIS is that we can't modify the data
conveniently. And the speed or portability is not idea enough. So we choose
Geopandas as the toolbox to manage the spatial data, and elasticsearch as the
query engine. Under this framework, we get to output multi-dimensional
results and easily control the data format.

Louis Ngamassi et al [5] choose table and line graph to show the statistics
based on twitter data. they use the classification template developed by
Bruns et al. [6] to map disaster-related tweet data so as to facilitate their
use for decision making . For our project, we hold the similar strategy to
show the result, but we use some graph generation tools to output
automatically. The advantage to do so is that we can focus more on the data
structure itself. The resulting field will come from some batch operations
form other field, which absolutely improves efficiency.

S. M. Mazharul Hoque Chowdhury et al[7] use word cloud graph to show high
frequency words. Moreover, they apply Naïve Bayes algorithm for the word
cloud. Then they get only the emotion class related words in the graph. We
also apply their strategy to our project with more classification criteria.
Since some high frequency words contain few information like the city name in
some topic. We add some removal instruction before using them. The other
things is that part of speech are also another important basis to do the
classification and affects the classification results of emotional words. So
we propose to use composite classification criteria as the basis for
displaying word cloud graphs.