\section{Conclusion}
\label{sec:conclude}
In this project, our team carried out a machine learning-based spatial
modeling of sentiment towards the pandemic. A rationale for this research
comes from the increased interest on analyzing a lot of issues related to
pandemic by using machine learning in academia. We have been pursued to
combine the ML and Deep Neural Network (DNN) with our spatial-data mining
technology so that this study contributes to the field by producing deeper
implications.

As a dataset for sentiment analysis the importance of Twitter dataset can not
be overstated. In recent days of DNN era, the public opinion expressed in
twitter became a valuable resource for numerous tasks including such
sentiment prediction. We paid attention on the applicability of DNN on the
Twitter dataset for the pandemic issue. By using some applications such as
Twarc and Elasticsearch we collected and stored such related dataset into our
own repository, and spatial datamining task such as querying is followed. By
doing such spatial query and filtering task, the target dataset could be
visualized into a map. Furthermore, we develop a sentiment classifier by
adopting the decision tree in Machine Learning. This module analyzes people's
opinion in tweeter data and classifies it into positive or negative, by
getting trained with pre-built sentiment score.

However, there are still another important task remained to achieve our team's
final goal. Since our current model simply mine the spatio-temporal dataset
and show how people's opinion is distributed differently by regions and time,
it requires to do further research to draw deeper implications for
contributing academia. Therefore, we plan to build a sub-topic clustering
module using Transformer and compare its result with our predicted sentiment
score. With adopting such a new NLP model architecture, it is expected to
track a distribution of public opinion on detailed issues, e.g., government
policy on social distancing, vaccine supply, etc., so that the study can have
practical implication to decision makers. 